%%%%%%%%%%%%%%%%%%%%%%%%%%%%%%%%%%%%%%%%%%%%%%%%%%%%%%%%%%%%%%%%%%%%
%% I, the copyright holder of this work, release this of into the
%% public domain. This applies worldwide. In some countries this may
%% not be legally possible; if so: I grant anyone the right to use
%% this work for any purpose, without any conditions, unless such
%% conditions are required by law.
%%%%%%%%%%%%%%%%%%%%%%%%%%%%%%%%%%%%%%%%%%%%%%%%%%%%%%%%%%%%%%%%%%%%

\documentclass[
  print, %% This option enables the default options for the
           %% digital version of a document. Replace with `printed`
           %% to enable the default options for the printed version
           %% of a document.
  Table,   %% Causes the coloring of tables. Replace with `notable`
           %% to restore plain tables.
  nolof,     %% `lof` Prints the List of Figures. Replace with `nolof` to
           %% hide the List of Figures.
  nolot,     %% `lot` Prints the List of Tables. Replace with `nolot` to
           %% hide the List of Tables.
  %draft, %TODO remove, place final instead
  11pt, % TODO disscus this
  oneside  %% twoside for print
  %% More options are listed in the user guide at
  %% <http://mirrors.ctan.org/macros/latex/contrib/fithesis/guide/mu/fi.pdf>.
]{fithesis3}
%% The following section sets up the locales used in the thesis.
\usepackage[resetfonts]{cmap} %% We need to load the a T2A font encoding
\usepackage[T1]{fontenc}  %% to use the Cyrillic fonts with Russian texts.
\usepackage[
  main=english, %% By using `czech` or `slovak` as the main locale
                %% instead of `english`, you can typeset the thesis
                %% in either Czech or Slovak, respectively.
  %german, russian, czech, slovak %% The additional keys allow
]{babel}        %% foreign texts to be typeset as follows:
%%
%%   \begin{otherlanguage}{german}  ... \end{otherlanguage}
%%   \begin{otherlanguage}{russian} ... \end{otherlanguage}
%%   \begin{otherlanguage}{czech}   ... \end{otherlanguage}
%%   \begin{otherlanguage}{slovak}  ... \end{otherlanguage}
%%
%% For non-Latin scripts, it may be necessary to load additional
%% fonts:
%\usepackage{paratype}
%\def\textrussian#1{{\usefont{T2A}{PTSerif-TLF}{m}{rm}#1}}
%%
%% The following section sets up the metadata of the thesis.
\thesissetup{
    date          = \the\year/\the\month/\the\day,
    university    = mu,
    faculty       = fi,
    type          = mgr,
    author        = Karel Kubíček,
    gender        = m,
    advisor       = {RNDr. Petr Švenda, Ph.D.},
    title         = {Advancing EACirc, an automation tool for cryptanalysis of stream, block and hash functions},
    TeXtitle      = {Advancing EACirc, an automation tool for cryptanalysis of stream, block and hash functions},
    keywords      = {randomness testing, cryptoanalysis, block functions, stream functions, hash functions, problem optimization, metaheuristics},
    TeXkeywords   = {randomness testing, cryptoanalysis, block functions, stream functions, hash functions, problem optimization, metaheuristics},
}
\thesislong{abstract}{
    TODO
}
\thesislong{thanks}{
    TODO (collegues, friends, Marta, Petr, family)
}
%% The following section sets up the bibliography.
\usepackage{csquotes}
\usepackage[              %% When typesetting the bibliography, the
  backend=biber,          %% `numeric` style will be used for the
  style=numeric,          %% entries and the `numeric-comp` style
  citestyle=numeric-comp, %% for the references to the entries. The
  sorting=none,           %% entries will be sorted in cite order.
  sortlocale=auto         %% For more unformation about the available
]{biblatex}               %% `style`s and `citestyles`, see:
%% <http://mirrors.ctan.org/macros/latex/contrib/biblatex/doc/biblatex.pdf>.
\addbibresource{thesis.bib} %% The bibliograpic database within
                          %% the file `example.bib` will be used.
\usepackage{makeidx}      %% The `makeidx` package contains
\makeindex                %% helper commands for index typesetting.
%% These additional packages are used within the document:
\usepackage{paralist}
\usepackage{amsmath}
\usepackage{amsthm}
\usepackage{amsfonts}
\usepackage{url}
\usepackage{menukeys}


\usepackage{cleveref}
\usepackage{wrapfig}
\usepackage{subcaption}
\usepackage{tabularx}

%% My own inputs:
% enabling new fonts support (nicer)
\usepackage{lmodern}
% code
\usepackage{minted}

\usepackage{tikz}
\usetikzlibrary{shapes,arrows,positioning}

% package to make bullet list nicer
\usepackage{enumitem}
\setitemize{noitemsep,topsep=3pt,parsep=3pt,partopsep=3pt}

% intendation
\usepackage{parskip}
\setlength{\parskip}{5pt}
\setlength{\parindent}{0pt}

\newcommand{\todo}[1]{TODO: \textit{#1}}
\newcommand{\rewrite}[1]{rewrite: \textit{#1}}
\newcommand{\reread}[1]{reread: \textbf{#1}}


\begin{document}


%% We will define several mathematical sectioning commands.
%\newtheorem{theorem}{Theorem}[section] %% The numbering of theorems
                               %% will be reset after each section.
%\newtheorem{lemma}[theorem]{Lemma}     %% The numbering of lemmas
%\newtheorem{corr}[theorem]{Corrolary}  %% and corrolaries will
                                %% share the counter with theorems.
%\theoremstyle{definition}
%\newtheorem{definition}{Definition}
%\theoremstyle{remark}
%\newtheorem*{remark}{Remark}

\chapter{Introduction}

Cryptography is a field of computer science, used daily by all Internet users. Users connecting to HTTPS websites are using asymmetric cryptography to generate a symmetric key for encryption of the following communication. For such use-case, as well as various others, are used cryptographic primitives as public-key encryption functions, hash functions, block functions and stream functions. Current research of cryptoprimitives has a nontrivial goal. Design and develop algorithms, such that can hold against unknown threats as long as possible. Therefore cryptographers follow the standardised approach in cryptoprimitives development to ensure quality by a lengthy process of preparation, comparison and testing.

\section{Cryptoprimitives selection process}

A standardisation committee reacts on inquiry for new or updated cryptoprimitive by issuing a competition for it. For example competitions during last 20 years were AES, which called for a symmetric block cipher, eSTREAM, for new stream ciphers, SHA-3 for hash function and CAESAR for authenticated ciphers. Those competitions last usually around five years. The first phase is collecting proposals, which are filtered in the second phase to the gradual selection of the winner candidate in the third. The winning solution became standard, and the developers will use it usually for at least ten years. For example, Data Encryption Standard was standardised in 1976, and Triple-DES modification surpassed it as a standard in 1999. Final successor Advanced Encryption Standard came in 2002, and it remains functional until today (2017).

\section{Cryptoanalysis}

During the second phase, proposals are analysed by cryptoanalysts. While proposing own design is possible for both skilled teams and almost cryptography beginners, the analysis is demanding process that needs both skill and time.

Cryptanalysis is expensive work, as the required cryptoanalyst skills are demanding. It consumes a significant amount of time to reject even proposals vulnerable to known threats. Therefore, a software tool capable of rejecting vulnerable proposals can save cryptoanalyst's time. Moreover, the cryptanalyst can use the tool for quicker dive into the candidates' specifications. Such tool can, for example, identify optional parts and show weak or even threat parts of the proposed function.

\subsection[Cryptorimitives properties]{Overview of expected properties of cryptorimitives}

Field of the analysis, which is this thesis aiming for, are hash functions, block ciphers and stream ciphers (further referred as functions). There are several properties, formulated as a minimal set of conditions for these functions which has to be satisfied. If the function does not meet some property, it is considered weak, while satisfaction of all of the properties does not imply nonvulnerable function. Some of the properties are following.

Irreversibility without the encryption key: from given ciphertext, the attacker is not able to reconstruct any bit of plaintext with a probability higher than 50\,\%. If the probability were over 50\,\%, the ciphertext would reveal some information about the plaintext.

Strict avalanche criterion: every bit flip of plaintext changes around 50\,\% of output bits. Not satisfying this would allow an attacker to start with plaintext of zeroes and continuously modify single bits to get closer to given ciphertext.

Oracle, semantic security: even thou plaintext usually does not look like a random data, the ciphertext have to look so. If the output were distinguishable from random data, it would reveal information about the plaintext. This property is tested in our tool EACirc. However, the previous properties are tested as well, yet not directly.

\section{Cryptoanalysis automatization}



\subsection[Optimization problem for a cryptoprimitive]{Building an optimization problem from property of cryptoprimitives}


\section{Statistical testing}

Testing of randomness is a common area of statistics research. Thus statisticians have developed a rigorous theory of probability, and they also produced practical tools, that can test randomness. These tools are used even in cryptography to analyse function for properties from the previous section.

\subsection{Statistical batteries}

The statistical battery is a set of simple tests, which are analysing simple properties of the binary stream. For example, a monobit test counts an amount of zeros and ones in the input data, which should be similar in the case of the uniform stream. The tests can analyse the stream for patterns, which should not be present in random data.

The design of the tests accords to some statistical property. As the theory behind the statistical test is complex, the development of the test is difficult. The test set is fixed, so they cannot adapt to the input. The main advantage of EACirc is an ability to adapt to tested data. The tool itself learns on given data, and after this learning process, the output is randomness test for this specific type of data.

Best known statistical batteries are (chronologically by time of development) NIST STS, Diehard and Dieharder and TestU01. While NIST STS is already completely surpassed by its successors, statisticians still consider it as a golden standard of statistical testing. The EACirc performs better than NIST STS, is comparable with Dieharder and currently perform worse than TestU01. The ultimate goal of implementation of metaheuristics would be surpassing Dieharder and compete with TestU01.

\subsection{EACirc}

\section{EACirc computation loop}

\section{EACirc's potential}

\subsection{Individual representation}

\subsubsection{Circuit}

\subsubsection{Polynomials}

\subsection{Speed-up of computation}

\subsection{Evaluation}

\subsection{Learning process}

\section{Problem optimization}

\subsection{Metaheuristics}

A computational problem is a task to find a feasible solution for given question. An optimisation problem is a task to find the feasible solution, which satisfies the criteria best. Usually, we formulate fitness function to compare how well are the criteria met. A heuristic is a problem specific technique for finding some an approximative solution, where exact techniques fail. Finally, the metaheuristic is technique abstracted from the problem so that it can be used for various optimisation problems.

Metaheuristics are applicable for optimisation problems if a programmer can provide three crucial parts. First is an instance of solution for the problem (despite the quality). Second is a function to vary the solution, called neighbourhood function. Third and usually the most important is the fitness function, which measures the quality of the solution. Metaheuristic uses these parts to follow the landscape of the problem, and they try to find better solutions during the time.

The main advantage of metaheuristics is that they are well known, compared together in many studies (see 849 references in Metaheuristics book [1]) that also provided tips for the implementation.

\subsection{ANN}

\subsection{...}

\chapter{Current approaches in problem optimization}

The main categorization of metaheuristics is by a number of solutions inspected in one moment: single-solution based metaheuristics and population-based metaheuristics. The original idea of EACirc used genetic programming -- a population-based metaheuristic for optimisation. One of the weaknesses of this approach was a rather difficult interpretation of the results [cite Martin's thesis - understanding "random" number 0.52].

The main advantage of having multiple concurrent solutions is handling population diversity. Diversity can help avoid local optimums and together with sexual crossover can also produce better individuals faster. However, maintaining the diversity of the population is difficult, if the fitness does not increase continuously [cite something - probably Martin's thesis]. An individual having a unique feature, which increases its fitness, is going to be spread trough mutations in next generations, rather than another feature would emerge randomly within a different individual. Besides, the implementation of sexual crossover was never widely used and tested [cite Martin's thesis].

The next solution proposed in [paper Syso] simplified the interpretation by using the population of only one individual. This solution allowed statistical postprocessing of fitness values and together with categories evaluator surpassed the previous approach in all tests [again Syso's paper?]. The population with a single individual changed the used method to local search. Because the test vectors were changed every 100 iterations, the metaheuristic can be described as iterated local search [cite], or sometimes called global search?.

EACirc 4.0 simplified and optimised the computations by substituting GALib -- a genetic algorithm library by a custom implementation of the iterated local search.  To question if iterated local search is the best approach, I analysed the problem and tried other metaheuristics.

\section{Single-solution metaheuristics}

Figure ... shows the classification of the improved variants of local search metaheuristics by authors of ParadisEO library.

\subsection{Local search}
\subsection{Iterative local search}
\subsection{GRASP}

The greedy randomised adaptive search procedure is splitting each iteration into two steps: a feasible solution construction by greedy algorithm and performing the local search using the generated solution. As we did not found any greedy algorithm for circuit construction that would reflect tested data, this approach was abandoned.

\subsection{Multi-start local search}

Is another approach how to surpass local optimum to find the global optimum. The metaheuristic starts the computation with k randomly generated individuals that are concurrently optimised by local search. However, this method is already used in EACirc, neither on the level of single computation, but in the iteration of experiment over 1000 runs per unique settings to obtain statistically significant results. Implementation of multi-start local search on the level of single computation would make the results interpretation more difficult, same as in the case of EA with a population in [Martin's thesis].

\subsection{Guided local search}

Guided local search is metaheuristic based on dynamic changing of the fitness function. If the metaheuristic got stuck in local optima, guided local search would change the fitness function so the individual can follow climbing.

The EACirc computation can get trapped in local optima mainly in two cases:

\begin{enumerate}
    \item the individual got lucky to guess current test set, but it will not be successful in next epoch, or
    \item the individual is already successful due to finding feature of the tested data, but it could not get better.
\end{enumerate}

We do not need to fix the first case, as such individual would be replaced in next epoch. However, we can use Guided local search to improve individual from the second case. Our alternative fitness function can be designed as the sum of our current fitness and e.g. the number of connectors. Such change would simplify the solution, which would help us with the interpretation of the last individual. Alternatively, we can use completely different fitness function.

\subsection{Noisy method}

The noisy method is a metaheuristic, which allows escaping local optima by noised fitness function. At the beginning of the computation, the noise is the strongest, and it disappears during the computation. The suggested approach for implementing the noise by [metaheuristics book, pp 160] is randomization of tested input. As EACirc works with discrete randomly-looking data, such approach could flip important bit which would make progress impossible. Therefore this method was abandoned.

\subsection{Smoothing method}

Smoothing method is the last inspected metaheuristics, which uses changes in fitness function to escape local optima. The method is based on smoothing the landscape of fitness function by computing the local average. It can work well for example figure [fitness function graph], however as the neighbourhood of an individual is enormous, and it is not continuous, this method would not be possible due to computation requirements.

\subsection{Variable neighbourhood search}

Compared to metaheuristics that change fitness function, variable neighbourhood search is not able to escape local optima in all the cases. However, it still can increase the escaping probability, as the searched neighbourhood space would contain more desired solutions. If we already have a successful solution, we can inspect only solutions with fewer connectors, or in contrast, we can press more to connect the individual in case that the input and output are not connected and fitness is equal to 0.

\subsection{Simulated annealing}

Remaining two single-solution metaheuristics are based on acceptance of non-improving neighbours. Short term acceptance of worse solution can lead to escape from local optima and finding global optima in whole computation.

Simulated annealing metaheuristic is inspired by the physical process of annealing in metallurgy. Annealing is an iterative process of heating the metal to increase the size of the crystals, which reduces the probability of a defect. During the process, the metal is heated to lower and lower temperatures.

Simulated annealing uses variable temperature, which describes probability for acceptance of worse neighbour. This probability is decreasing during the computations, such that in the beginning, the likelihood of acceptance worse solution is high, but at the end of the computation, only better solutions are accepted.

This metaheuristic is widely used, which was the reason for implementing it as a testing scenario for metaheuristics. On the other hand, accepting non-improving solution is has a very limited positive effect on computation of EACirc, as a bad neighbour can substitute a good solution.

\subsection{Tabu search}

Like Simulated annealing, Tabu search can also accept non-improving solutions to escape from local optima. However, the decision whether to accept a non-improving solution is not based on probability, but on knowledge, if considered solution was visited. Therefore, Tabu search holds a memory of visited solutions, called \textit{tabu list}, and it would not use twice the same solution. This way, Tabu search avoids cycles during the optimisation process.

Application of Tabu search in EACirc has several issues. Firstly, how to store used individuals (memory usage may be a problem), which can be solved by storing only a hash of the solution. Secondly, the neighbourhood space of an individual is huge [note calculation] -- higher than the number of individuals tested during whole computation. Choosing twice the same solution is unlikely. Therefore I chose to try only Simulated annealing and leave the decision of Tabu search implementation on the result of Simulated annealing.

\section{Multi-solutions metaheuristics}
\subsection{Evolutionary algorithms}
\subsection{Ant colony optimization}
\subsection{Scatter search}
\subsection{Swarm intelligence}
\section{Other methods}
\subsection{ANN}
\subsection{...}



\chapter{Experiment methodology}

The main testing goal of EACirc is to asses randomness of ciphertext of cryptographic functions. As most of the cryptographic functions are semantic secure, EACirc is able to identify non-randomness only for reduced functions. The functions can be reduced in number of rounds, by processing selected plaintexts, and by selecting only specific bits of the output. As analysis of reduced versions of cryptographic functions is the main goal of EACirc project, it is also selected as the testing scenario for comparison of analysed metaheuristics.

The approach for evaluation is \todo{[briefly?] described in [chapter EACirc]}. The tested data were produced from 16 different cryptographic functions and pseudo random data generator PCG [cite] as reference data. These functions are reduced to the edge of distinguishing capability of EACirc, such that the EACirc is capable to distinguish function reduced to $i$ rounds, but it is not able to distinguish $i+1$ rounds\footnote{Further is this round settings reffered as "an edge".}. Usually, some extra surrounding is tested as well for broader usage of the results.

\section{Used data}

Previous works on EACirc analysed candidates of cryptoprimitives competitions [cite eSTREAM, SHA-3 and CAESAR works]. These works extended EACirc by many functions sharing the same interface. This approach was effective for broad analysis of the competition candidates, but it does not test well known cryptographic functions like AES and DES.

Therefore, for the comparison with other randomness testing tools and other research papers on semantics security, we decided select new test set of cryptographic functions. We chose the finalist of competitions SHA-3 (BLAKE, Grostl, JH, Keccak and Skein) and eSTREAM (HC-128, Rabbit, Salsa20, SOSEMANUK and Grain), and above that, we implemented other block (AES, DES, 3-DES and TEA) and stream (RC4) functions.

\subsection{Reimplementation of stream handling in EACirc}

The implementation of SHA-3 and eStream candidates in EACirc was called projects. The projects were handling everything related to data generation: generation of IV, key and plaintext, produced by specific sources. Every project has its own implementation of counter plaintext generator, strict avalanche criterion generator and so on.

For implementation of the new project for block ciphers, we decided to reimplement data generation in more variable and modifiable way. For this, we decided for data manipulation via streams, whose can be pipelined into each other. Each stream implements simple interface with constructor and function \texttt{next}, which returns next test vector. The stream is specified only by the output size of the test vector.

The whole data generations is currently provided by streams. Firstly, all the data sources are streams, from true and false bits generators, over counter and strict avalanche criterion test vector generators, to PRNG test vector generators. Secondly, the projects eSTREAM, SHA-3 and the block ciphers take these simple streams as the input for computation. These projects are streams as well, as they produces ciphertext vectors. And finally, the ciphertext can be postprocessed by streams for selection of only some bits of the output.

All this described data manipulation and project eSTREAM and SHA-3 refactoring was implemented as a part of this thesis.

\subsection{Implementation of functions}

As the source of AES, DES and 3-DES implementations was used educational repository~\cite{cryptoFunc}. The author of this repository verifies the encryption to the test vectors to ensure correctly working implementation. The only modification of the code is limitation of the functions in the number of rounds.

TEA implementation is based on Wikipedia code~\cite{teaWiki}, and it was already implemented in EACirc before. The implementation was simplified and reduced due to changes in the interface. It is also reduced to selected amount of rounds.

RC4 is also based on the educational repository~\cite{cryptoFunc}, but as RC4 has no round structure, therefore, the code is unchanged.

All these functions together with the new interface of new block project and test vectors processing algorithm were released in minor version EACirc 4.1.

\subsubsection{?Testing correctness of the solution?}
\subsubsection{?Publication of the generator tool?}


\section{Approach specific methodology}

As the tested scenarios vary not only by method, but by used tools as well, the methodology changes for given metaheuristics. The base of the experiment was used for single-solution metaheuristics, as they were implemented directly to the EACirc and they inherited the experiment environment of our tool.

For direct inter-method comparison, we need to use the same data for testing. Therefore, all used data from the \todo{Usable testbed} are generated from the same implementation. This was allowed by implementation of the Generator tool \todo{described later in ... reference to part of the thesis with Generator}.

\subsection{Single-solution methodology}

\subsection{Statistical evaluation}

The fitness of an individual is a $p$-value of $\chi^{2}$ test on output bytes for current test vectors set. For a good individual for a weak version of function, the fitness is close to 1. As the quality of the individual decreases, the fitness decreases as well. However, completely random guess like coin tossing has the $p$-value uniformly distributed on interval (0, 1).

Getting one such $p$-value for whole computation says nothing, as we could be lucky to observe high, however internally random, fitness. Therefore we perform Kolmogorov-Smirnov test of uniformity of the $p$-values. This test is using critical value $\alpha = 1\,\%$, therefore it rejects the null hypothesis that tested data seems nonrandom in 1\,\% of runs. To avoid such case, we run the experiment with the same settings 1\,000 times and we analyse the rejection ration given as amount of failed KS tests over 1000. If it is around 1\,\%, the data seem to be random, while in it is much higher, the experiment rejected randomness hypothesis of tested data.

This approach is vastly increasing the amount of data needed for the computation, as well as the overall run-time. This approach is a trade-off for success rate of EACirc, as different evaluators do not need statistical evaluation, hence they need much less data and they are slightly worse based on results from~\cite{svenda2013towards}.

When we consider basic setup for single-solutions metaheuristics, EACirc consumes 2.4\,GB of ciphertext of tested function~\cref{fig:dataUsage}. The runtime of the experiment on Metacentrum varies from 4 to 20 hours on single core.

%http://tex.stackexchange.com/questions/57732/placing-figures-inside-a-two-column-document 3rd answer
\begin{figure*}[t]
    \begin{equation*}
        \begin{aligned}
    \Sigma = 1\,000 \;
             \frac{\textnormal{runs}}
                  {\textnormal{experiment}}
             \cdot
             \left(
             \frac{30\,000 \frac{\textnormal{generations}}{\textnormal{run}}}
                  {100 \; \frac{\textnormal{generations}}{\textnormal{test set}}}
             \cdot
             1\,000 \;
             \frac{\textnormal{vectors}}
                  {\textnormal{test set}}
             \cdot
             8 \;
             \frac{\textnormal{bytes}}
                  {\textnormal{vector}}
             \right) \approx\\
             \approx 2,4 \textnormal{\,GiB per experiment}
        \end{aligned}
    \end{equation*}
    \caption{The amount of data analyzed by EACirc for a single configuration of randomness testing experiment.}
    \label{fig:dataUsage}
\end{figure*}


\section{Computation automatization}

As EACirc needs repeated computation on Metacentrum grid infrastructure, the automation for binary deployment, computation distribution, aggregation of the results and their postprocessing are needed.

The scripts consist of a preprocessor, which creates necessary configuration files for the experiment, scripts for distribution among the computation infrastructure and following results collection, postprocessing script, that will aggregate the results, and the final script for generation of human readable version of the script as well as database dump of the experiment\cref{fig:scripts}.

Earlier was in use solution called Oneclick created by Lubomír Obrátil for his bachelor thesis~\cite{obratilBcThesis}. The Oneclick was a single tool for both preprocessing and postprocessing phase, and the run was parametrized. However, this approach is outdated for MetaCentrum infrastructure. The advantages of new scripts are: 

\begin{itemize}
    \item they are one purpose, so it is easier to modify one without affecting other,
    \item they are written in bash (running scripts for MetaCentrum scheduler) and Python (processing), so they do not need to be compiled, and more researchers understand, how to modify them,
    \item they are mainly written for the newest released EACirc version (but they support EACirc 3 as well), so they fit more the computation process of new version, so they save a vast part of processing,
    \item they are faster (up to 3\,600 times in case of postprocessor), as MetaCentrum computation do not require previously used redundancy procedures.
\end{itemize}

These scripts are expected to be modified by individual researchers for the needs of their experiments, so the scripts are stored in GitHub repository~\cite{eaUtils}, and users are supposed to fork them. The repository also contains \texttt{readme.md} file, with specifies the first steps with computations on MetaCentrum with these scripts. Every script then contains documentation in the code (concerning expected modifications for individual purposes), and when it is run without arguments, it prints out a help message.

\begin{figure}[H]
    \centering
    \tikzstyle{block} = [draw, rectangle, minimum height=3em, minimum width=3em]
    \tikzstyle{el} = [draw, ellipse, minimum height=3em, minimum width=3em]
    \tikzstyle{cl} = [cloud, cloud puffs=10,cloud puff arc=120, minimum height=3em, minimum width=3em]
    \begin{tikzpicture}[>=stealth, auto, node distance=3cm, every text node part/.style={align=center}]
        \node [el] (e) [] {\bf{experiment idea}};
        \node [block] (pre) [below of = e,yshift=0.5cm] {\bf{preprocessor}};
        \node [block,text width=3.5cm] (run) [below of = pre,yshift=-0.5cm] {\vspace{0.1cm}{\bf runner scripts} \texttt{run\_experiment.sh} runs for every config file script \texttt{single\_job.sh}};
        \node [block] (eac) [below of = run] {EACirc};
        \node [block] (pos) [below of = eac,yshift=0.6cm] {postprocessor};
        \node [block] (gen) [below of = pos,yshift=0.6cm] {table generator};
        \node [el] (res) [below of = gen,yshift=0.6cm] {readable results};

        \draw [->] (e) -- node [right] {researcher modifies\\general setting} (pre);
        \draw [->] (pre) -- node [right] {produces config for\\all combinations\\of function and round} (run);
        \draw [->] (run) -- node [right] {distributes computations\\on MetaCentrum grid} (eac);
        \draw [->] (eac) -- node [right] {results are returned\\to the frontend server} (pos);
        \draw [->] (pos) -- node [right] {aggregates the results\\and computes the statistics} (gen);
        \draw [->] (gen) -- node [right] {generates human readable tables\\and dump the results} (res);
    \end{tikzpicture}
    \caption{Execution of EACirc on MetaCentrum grid -- order and purpose of the scripts.}
    \label{fig:scripts}
\end{figure}

\subsection{Preprocessor}

EACirc computation is based on a config file, and that specify what are the sources of data for comparison and various other settings. As the test suite contains tens variations of parameters, it is necessary to automatize config file creation, so the user would not make a mistake during this process.

The scripts are written in Python and have dedicated version per the scenario of tested functions.

\subsection{Running scripts}

The scripts for distribution of computations among the MetaCentrum network are mainly work of Martin Ukrop for the purpose of his thesis~\cite{ukropMgrThesis}. However, they are continuously maintained in \texttt{EACirc-utils} repository~\cite{eaUtils} for current EACirc version.

As each experiment needs to be run multiple times for statistical evaluation, the scripts repeat the computation in a loop with different seeds usually for 1\,000 times. All runs are then collected by running scripts.

The current version of these scripts is also published on the GitHub~\cite{eaUtils}, and the forked repository is ready to run experiments in the usual way.

\subsection{Postprocessor}

As the computations are repeated for statistical significance, the postprocessor script has to aggregate results from many produced files. It searches through output files and collects the results. Afterwards, it calculates an aggregated statistics of the experiment.

The output of the postprocessor is \todo{\texttt{JSON} dump} of the results, whose are used both for human readable results generation and also storing the results for later usage.

\subsection{Table generator}

The generator takes as an input the \texttt{JSON} dump of the results from the postprocessor. Currently, it only generates \LaTeX table with highlighted results, but it can be rewritten for more complex manipulation with the results.

\subsubsection{Goal specification}

The ultimate goal of the experiments with metaheuristics is to increase amount of rounds distinguished by EACirc. However, such goal is enormously difficult, as the randomness increases rapidly with the additional rounds. We would need functions with low increase of entropy per round as a benchmark, but we did not observed such behaviour by any of the tested functions, nor our own design of benchmarking function do not behave that way [LUT cipher].

Less ambitious, but still useful goal can be increase of the rejection rate of the null hypothesis per experiment. Rejecting only 10\,\% of runs imply having difficulties with analysis of the output test. The higher is the rejection rate, the easier is interpretation of found distinguisher. Also in case our rejection rate is close to critical value $\alpha$, then the higher rejection rate can overcome the edge value\footnote{For 1000 runs per experiment, this edge value is typically $2\cdot\alpha$. Such results can happen in less than 0.15\,\% of the experiments.} and we can reject the overall null hypothesis about the randomness of the source with higher confidentiality.

Other goal can be decrease of the amount of computations or the amount of used data. As we usually fix the epoch length to 100 iterations, the data usage and CPU time are directly related. And as the amount of used data is already high in this settings of EACirc, the time comparison is more valuable than consumed data comparison.

We can also analyse specific goals per metaheuristics. Both Variable neighbourhood search and Guided local search set complementary target in simplification of the circuits for later analysis. This goal is more difficult to measure, so the analysis is done less rigorously.

EACirc project is using adaptive learning. During the computation, the tested data are changing. We analysed over-learning of the individuals during the learning process, and the over-learning can be analysed as well. Therefore, decrease over-learning can be another goal for metaheuristics.

\subsection{Multi-solutions methodology}
\subsection{ANN methodology}

\chapter{Experiment results}

To verify the implementation, we perform sanity experiment comparing random data to random data. Such test gives us real measured expected output in case that two tested streams cannot be distinguished by selected approach. Expected value of such experiment differs per methodology and therefore is stated per every testing scenarion.

\section{Single-solution}

%http://stattrek.com/online-calculator/binomial.aspx

Single solutions metaheuristics use evaluation described in [reference to Statistical evaluation subsubsect]. As we use critical value $\alpha=1\,\%$, the random random experiment should reject randomness hypothesis in around $1\,\%$ of the runs. As we would like to have better confidentiality in this sanity experiment, we perform not only 1\,000 runs for this settings, but 100\,000 runs. Confidentiality for such run can be computed from binomial distribution and obtaining value $1.2\,\%$ has probability 4.787\,\% for 1\,000 runs, while it is lower than 0.001\,\% for 100\,000 runs.

Other results are summarized in table, where every row corresponds to one cryptographical function and this function is reduced to rounds given by column of the table. The dash means that such setup was not tested. Number represents ratio runs where KS test rejected the randomness hypothesis by 1000 runs. Values around measured ratio from random-random experiment are expected, results twice this value can happen in less then 0.15\,\% measures for randomly looking data. Therefore results over double of this measured critical value are considered as nonrandom.

Our own evaluation of the result is highlighted by color. Red highlight sign that EACirc detected nonrandomness. Green highlight sign that EACirc has not found anything and from the notion of hypothesis testing, we cannot say anything about the randomness of data (we cannot accept the randomness hypothesis).

\subsection{Iterated local search baseline}



\subsubsection{Overlearning analysis}
\subsection{Simulated annealing}
\subsubsection{Overlearning analysis}
\subsection{Guided local search}
\subsubsection{Overlearning analysis}
\subsection{Variable neighbourhood search}
\subsubsection{Overlearning analysis}
\section{Multi-solutions}
\subsection{Bruteforce baseline}
\subsection{Ant colony optimization}
\section{ANN}
\section{Inter-approach comparison}

\chapter{Related work}

\section{Statistical batteries}
\subsection{Results}
\section{?Search for papers on same topic?}
\section{EACirc previous work}

\chapter{Conclusion}

\section{Methodology}
\section{Results}
\section{Future work}


\end{document}
